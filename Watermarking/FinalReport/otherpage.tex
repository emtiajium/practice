%\documentclass[12pt,a4paper]{report}
%\usepackage{blindtext}
%\usepackage{graphicx}
%
%\usepackage{hyperref} % this for link


%\begin{document}

\pagenumbering{roman}

%page 1

\thispagestyle{empty}

\begin{center}
\textbf{{\large Bachelor of Science in Computer Science and Engineering}}

\vspace{75mm}

\textbf{{\large A Zero-Watermarking Scheme Based on Discrete Hartley Transform for Audio Signal} }

\vspace{30mm}

\textbf{Emtiaj Hasan}

\textbf{ID: 1004050}

\vspace{15mm}

\textbf{March, 2016}

\vfill

\textbf{{\large Department of Computer Science \& Engineering} \newline
{\normalsize Chittagong University of Engineering \& Technology} \newline
{\small Chittagong-4349, Bangladesh}}
\end{center}

\newpage

%page 2

\thispagestyle{empty}

\begin{center}
\textbf{{\large A Zero-Watermarking Scheme Based on Discrete Hartley Transform for Audio Signal} }

\includegraphics[scale=.75]{image/cuet-logo.png}


This thesis is submitted in partial fulfillment of the requirement for the degree of Bachelor of Science in Computer Science \& Engineering. \newline

\vspace{10mm}

Emtiaj Hasan

ID: 1004050


\vspace{20mm}

Supervised by \

Dr. Pranab Kumar Dhar \

Assistant Professor \

Department of Computer Science \& Engineering (CSE) \

Chittagong University of Engineering \& Technology (CUET)




\vfill
\textbf{{\large Department of Computer Science \& Engineering} \\
{\normalsize Chittagong University of Engineering \& Technology} \\
{\small Chittagong-4349, Bangladesh}}
\end{center}

\newpage

%page 3

\thispagestyle{empty}

The thesis titled \textbf{\textquotedblleft A Zero-Watermarking Scheme Based on Discrete Hartley Transform for Audio Signal\textquotedblright} submitted by Roll No. 1004050, Session 2013-2014 has been accepted as satisfactory in fulfillment of the requirement for the degree of Bachelor of Science in Computer Science \& Engineering (CSE) as B.Sc. Engineering to be awarded by the Chittagong University of Engineering \& Technology (CUET).

\vspace{20mm}
\begin{center}
{\Large \textbf{Board of Examiners}}
\end{center}

\vspace{10mm} 
\noindent 1. \rule{4cm}{0.4pt} \hfill Chairman  \newline
Dr. Pranab Kumar Dhar \newline
Assistant Professor \newline
Department of Computer Science \& Engineering (CSE) \newline
Chittagong University of Engineering \& Technology (CUET) \newline


\vspace{10mm} 
\noindent 2. \rule{7cm}{0.4pt} \hfill Member  \newline Professor Dr. Mohammed Moshiul Hoque \hfill (Ex-officio) \newline
Head \newline
Department of Computer Science \& Engineering (CSE) \newline
Chittagong University of Engineering \& Technology (CUET) \newline

\vspace{10mm} 
\noindent 3. \rule{3.5cm}{0.4pt} \hfill Member  \newline
Md. Monjur-Ul-Hasan \hfill (External) \newline
Assistant Professor \newline
Department of Computer Science \& Engineering (CSE) \newline
Chittagong University of Engineering \& Technology (CUET) \newline

\newpage


%page 4

\thispagestyle{empty}

\begin{center}
{\Large \textbf{Statement of Originality}}
\end{center}

\vspace{10mm}
It is hereby declared that the contents of this thesis is original and any part of it has not been submitted elsewhere for the award of any degree or diploma.

\vspace{50mm}

%\noindent\rule{6cm}{0.4pt} \hfill \rule{6cm}{0.4pt}
%
%\noindent\textbf{Signature of the Supervisor}	 \hfill \textbf{Signature of the Candidate}	\newline
%\textbf{Date}:  \hspace{77mm} \textbf{Date}: \newline

\noindent\rule{5.7cm}{0.4pt}

\noindent\textbf{Signature of the Candidate} \newline \textbf{Date}:

\newpage

\section*{Acknowledgment}
This thesis gives me an opportunity to thank all of the people who have helped me throughout my graduation life. First of all, I am grateful to my honorable project Supervisor Dr. Pranab Kumar Dhar, Assistant Professor, Department of Computer Science and Engineering, Chittagong University of Engineering and Technology, for the guidance, inspiration and constructive suggestions which were helpful in the preparation of this project. I also convey special thanks and gratitude to Professor Dr. Mohammed Moshiul Hoque, honorable head of the Department of Computer Science and Engineering, Chittagong University of Engineering and Technology, for his kind advice. I would also like to extend my gratitude to all of my teachers for their valuable guidance in every step of my learning stage. I would like to thank my friends for their cooperation that has helped in the successful completion of the project. Special thanks to the staffs of the department for their assistance. Last but not the least, I would like to thank my parents for supporting me throughout my entire life. Without their encouragement, it would not be possible to make this achievement.

\newpage

\section*{Abstract}
Internet is the fastest medium of transferring data to any place in a world and a popular digital media is audio. Various cloud-based storage contains a huge database of digital audio. Moreover, for the cheaper price of different storage device e.g., flash drive, make it easier for people to share, distribution of audio files easily and sometimes illegally. For these reasons, the chances of copyright infringement are higher than ever and measures against piracy were never so demanded. In this view, an audio watermarking field is very important. Considering the growing demand for this field, this thesis proposes a new zero-watermarking scheme based on Discrete Hartley Transform (DHT) for audio signal. In this scheme, DHT is performed on audio and a binary pattern is generated so that it can be used for extraction of the watermark in later. The experimental results show that this algorithm can resist various attacks. In most of the cases, the correlation between the original watermark and the extracted watermark is more than 0.9 and also bit error rate is less than 12\% which demonstrates that the proposed method is a suitable candidate for copyright protection. Furthermore, comparing with other existing schemes, it shows much better robustness against several attacks.

%\end{document}